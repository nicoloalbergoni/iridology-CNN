Dato che le diverse funzioni nel codice contengono svariati parametri di configurazione si è pensato di creare un apposito file contenente le associazioni parametro-valore, chiamato \texttt{config.ini}. Il file è strutturato in sezioni ed ogni sezione ha associato un insieme di parametri ad essa relativi; questi rappresentano i parametri fondamentali di tutte le principali funzioni presenti nel programma. Lo scopo di questo file è creare un metodo più rapido e semplice per la modifica dei valori dei parametri, in questo modo un utente che vuole effettuare test diversi, non necessariamente riguardanti l’iridologia, può semplicemente andare a modificare i valori dei parametri di interesse senza dover modificarli direttamente nelle chiamate alle funzioni nel codice.
