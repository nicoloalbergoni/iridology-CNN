Per la realizzazione dell’intero progetto si è utilizzato come linguaggio di programmazione Python 3. Si è scelto questo linguaggio per diversi motivi, il primo in assoluto è relativo al fatto che python rappresenta il linguaggio maggiormente utilizzato nello sviluppo di applicazioni legate al mondo del Machine Learning, difatti i più moderni framework per l’apprendimento automatico, come Tensorflow ad esempio, sono sviluppati in python. Tuttavia, questo non è l’unico motivo, anzi, ci sono altre ragioni che hanno portato alla scelta di questo linguaggio, ad esempio altri punti di forza di python sono la semplicità, la flessibilità e la versatilità, è infatti disponibile sulle principali piattaforme in circolazione (Windows, Linux e Mac OS). Altre caratteristiche sono il supporto di più paradigmi, tra cui: object oriented, strutturale e funzionale e la grande disponibilità di moduli pronti all’uso che permettono di arricchire il set di funzioni base del linguaggio. Tra i diversi moduli utilizzati nell’applicativo i principali sono:
\begin{itemize}
  \item \textbf{Python - OpenCV (cv2)}:  OpenCV sta per Open Source Computer Vision Library; Python - OpenCV è la versione python della popolare libreria utilizzata per la visione artificiale. Nel progetto viene utilizzata per le varie funzioni di elaborazione delle immagini
  \item \textbf{Numpy}: questa libreria aggiunge al linguaggio il supporto di array multidimensionali e matrici anche di grandi dimensioni; inoltre aggiunge anche molte funzioni matematiche di altro livello utili per operare con questo tipo di dati
  \item \textbf{Tensorflow}: framework sviluppato per l’apprendimento automatico. Contiene una vasta gamma di funzionalità legate al mondo del Machine Learning, fornisce moduli ottimizzati per la realizzazione di algoritmi o modelli di ML. Nell’ applicativo viene utilizzato soprattutto per la creazione della rete neurale
  \item \textbf{ConfigParser}: libreria che implementa un file parser per gestire file di configurazione strutturati
  \item \textbf{os}: modulo nativo di python che permette di utilizzare funzioni del sistema operativo. Ad esempio nel progetto viene spesso utilizzato per fare controlli su file, creare/eliminare cartelle e gestire path
\end{itemize}

Per quanto riguarda l’esecuzione dei vari script si è scelto di utilizzare un ambiente Python virtuale isolato, in modo tale da separare le dipendenze del progetto dalle altre dipendenze degli altri ambienti. A tale scopo si è utilizzato il tool \texttt{Virtualenv} che permette appunto la creazione di un interprete Python nella directory locale, separato dagli altri interpreti presenti nel sistema. Grazie a questo strumento si possono installare librerie nell’ambiente locale senza andare a creare potenziali conflitti con librerie appartenenti ad altri ambienti. Per la creazione dell’ambiente virtuale basta lanciare il comando \texttt{virtualenv venv} nella directory del progetto, in questo modo si va a creare una copia dell’interprete Python di default del sistema. Successivamente si attiva l’ambiente virtuale tramite il comando \texttt{source venv/bin/activate} (Per Window basta eseguire lo shell script \texttt{activate} presente in \texttt{venv/Scripts} ), si può poi procedere all’installazione delle dipendenze. Il file \texttt{requirements.txt} contiene la lista delle dipendenze per il progetto, per installarle basta lanciare il comando \texttt{pip install -r requirements.txt}; al termine dell’installazione le varie librerie saranno presenti unicamente nell’ambiente locale. Dopo l’installazione delle dipendenze si potranno eseguire gli script semplicemente chiamando il comando \texttt{python script.py}.
