\thispagestyle{plain}
\chapter*{Abstract}

Il continuo sviluppo delle tecnologie di apprendimento automatico ha reso più semplice ed efficace il monitoraggio dei dati relativi alla salute dell’uomo. L’obiettivo di questa tesi è quello di estrarre ed elaborare informazioni a partire da immagini dell’iride al fine di poterle utilizzare come strumento diagnostico. Il software è composto da due parti, la prima legata all’elaborazione dell’immagine, in particolare applicazione di filtri adeguati e metodi di riconoscimento, segmentazione e scaling dell’iride. La seconda parte rielabora i dati ricavati precedentemente al fine di poterli utilizzare in opportuni algoritmi di machine learning. L'implementazione proposta utilizza come strumento di apprendimento automatico una Convolutional Neural Network (CNN) per studiare la veridicità dell’iridologia, una tecnica di diagnosi che si basa sull’analisi di sezioni dell’iride.
