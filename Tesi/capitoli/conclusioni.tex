Grazie a questo elaborato è stata approfondita la conoscenza dei vari metodi di elaborazione delle immagini, comprendendo la grande importanza della loro implementazione al fine di ottenere degli input ottimali per una rete neurale. Sono stati esplorati infatti diversi filtri in ambito biometrico, imparando ad utilizzare la libreria OpenCV per la loro strutturazione e applicazione.

Si sono compresi inoltre il funzionamento e la struttura delle Convolutional Neural Networks, branca del machine learning in continua evoluzione. In particolar modo sono state studiate le diverse funzioni dei layer che costituiscono un modello di una CNN, imparandone il funzionamento tipico in ambito di computer vision per un caso di classificazione binaria. La rete implementata è idealmente buona per un problema di questo tipo, ma i risultati ottenuti non sono stati considerati in quanto non si è potuta verificare l’effettiva validità dell’iridologia, in assenza di dati “reali” per il training della rete neurale. Ciononostante il progetto è aperto a futuri ampliamenti in tale ambito (e non solo), in quanto con i giusti dati si può effettivamente verificare la veridicità di questa teoria medica alternativa.
