Dopo il caricamento delle immagini le funzioni che vengono richiamate sono \texttt{pupil\_recognition} e \texttt{iris\_recognition}, sono le funzioni che effettivamente si occupano di individuare la pupilla e l’iride dalla foto dell’occhio. Al fine di ottenere un maggior numero di riconoscimenti è stato necessario implementare diversi filtri che preparassero l’immagine al vero e proprio algoritmo di riconoscimento, la funzione che li applica si chiama \texttt{filtering} e viene invocata come prima elaborazione di \texttt{pupil\_recognition} e \texttt{iris\_recognition} in egual modo. Il primo passo fondamentale è trasformare l’immagine da BGR a grayscale tramite la funzione \texttt{cv2.CvtColor} di OpenCV, in quanto i colori diversi dalla scala di grigi risultano ininfluenti nell’identificazione della pupilla e dell’iride. Successivamente si procede all’applicazione di due trasformazioni morfologiche.
