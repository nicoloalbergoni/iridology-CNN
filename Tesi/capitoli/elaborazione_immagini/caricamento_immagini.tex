La prima funzione è \texttt{load\_image}, si occupa di caricare le immagini presenti nel path passato come parametro e le restituisce come un array di immagini (matrici di pixel). Il path è preimpostato e punta alla cartella \texttt{DATA\_IMAGES} che, come già accennato in precedenza, contiene il dataset di immagini CASIA divise per tipologia. Il caricamento viene reso possibile dalla funzione \texttt{cv2.imread} della libreria OpenCV che carica le immagini come matrici di pixel codificate nel formato BGR (Blue Green Red). Qualora fosse necessario è possibile abilitare a \texttt{True} il parametro \texttt{RESIZE} presente nella sezione \texttt{PREPROCESSING} del file di configurazione,in questo modo viene effettuato un resize automatico sulla base della dimensioni specificate dal parametro \texttt{RESIZE\_SHAPE}, nel nostro caso non è stato necessario fare un resize in quanto tutte le immagini erano già della stessa dimensione (320 x  280). La funzione ritorna l’array di matrici a tre livelli contenenti i valori BGR dei pixel delle immagini e l’array contenente i nomi dei file delle immagini.
