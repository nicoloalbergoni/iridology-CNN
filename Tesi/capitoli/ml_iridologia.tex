Nel seguente progetto si è cercato di utilizzare un approccio alternativo alle classiche ricerche mediche per dare un parere relativo alla veridicità o meno dell’iridologia. Si è pensato di utilizzare il Machine Learning: esso è lo studio scientifico di algoritmi che creano modelli matematico-statistici, basati sull’inferenza e la ricerca di pattern nei dati, affinché una macchina possa effettuare previsioni senza che venga esplicitamente programmata per farlo. L’accuratezza di tali previsioni è legata al processo di ricerca dei pattern sopracitato, detto anche processo di training: il modello apprenderà le variazioni tra questi dati di training e le utilizzerà per capire quale previsione dare in output a fronte di un nuovo dato. La precisione di queste previsioni sarà quindi strettamente legata ai dati su cui si fa training del modello in riferimento alla quantità e qualità di essi. L’approccio alternativo si basa sull’utilizzo di un modello, opportunamente accurato, che sia in grado, partendo da un segmento di iride legato ad una parte del corpo secondo la mappa dell’iridologia, di produrre in output una predizione riguardo la presenza o meno di un problema relativo alla parte del corpo associata al suddetto segmento. In questo modo si può confrontare il risultato prodotto dal modello con il parere di un iridologo; se nella maggior parte dei casi i pareri risultano discordanti significa che, con buona probabilità, non c’è una correlazione diretta tra iride e patologia. Tuttavia è doveroso fare alcune premesse: il modello utilizzato, al fine di produrre previsioni il più possibile corrette, deve avere una buona accuratezza; inoltre, per garantire la validità delle previsioni, è necessario fare training su immagini che abbiano una vera validità medica. Ad esempio: se si vuole cercare il riscontro di un problema al cuore sull’iride, bisogna fare training con segmenti relativi alla sezione dell’iride dedicata al cuore sia di persone di cui si conosce per certo l’esistenza di un problema al cuore sia di persone che non hanno nessun problema all’organo. In questo modo il modello imparerà a capire se vi è una relazione tra porzione di iride e problema al cuore. Una predizione prodotta dal modello ed una valutazione fatta da un iridologo possono poi essere confrontata con il parere di un medico professionista per avere un riscontro oggettivo. Per poter ottenere questo obiettivo si è ricercato un modello che fosse in grado di analizzare immagini al fine di capirne i pattern. Il modello più idoneo per questo tipo di problema è quello delle Convolutional Neural Network (CNN), una sottoclasse di modelli di machine learning dette Artificial Neural Network. Una descrizione più dettagliata delle CNN verrà trattata nel capitolo X.
