Nell’ambito del deep learning, una convolutional neural network (CNN) è una branca delle deep neural networks, applicata più comunemente all’analisi e la riconoscimento di pattern nelle immagini. Le CNN sono ispirate da processi biologici nei quali il pattern di connettività dei neuroni somiglia a quello della corteccia visiva animale. Singoli neuroni rispondono solo a stimoli provenienti da una specifica regione del campo visivo. 

Una CNN è progettata in modo tale da utilizzare al minimo la pre-elaborazione, in quanto la rete impara da sola a gestire i suoi filtri e la loro applicazione, cosa che non può avvenire in altri algoritmi di riconoscimento delle immagini; le informazioni che passano per la rete influenzano la struttura stessa della CNN poiché una rete neurale cambia, o impara, in base agli input e agli output intermedi. Una CNN consiste infatti in un input layer, un output layer e molteplici layers intermedi, chiamati hidden layers. Questi ultimi possono essere di vario tipo, tipicamente per una CNN vengono utilizzati layers convoluzionali, layers di attivazione (es. RELU), pooling layers, dense layers e flatten layers. 
