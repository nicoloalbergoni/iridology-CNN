Negli ultimi anni l’analisi dell’occhio si è rivelata particolarmente interessante in diversi ambiti, quali sicurezza, scanner biometrici, identificazione di persone, medicina, etc. In particolar modo l’iride è la porzione dell’occhio che contiene la maggior parte delle informazioni utili a fini scientifici. In questo progetto di tesi si esplorano le diverse tecniche di elaborazione di immagini dell’occhio al fine di individuare ed estrapolare regioni di interesse dell’iride (settori di iride chiamati ROI). Lo scopo finale è quello di utilizzare queste regioni come dato fondamentale per effettuare analisi di vario tipo con un approccio alternativo; tuttavia di per sé le zone d’interesse trovate non risultano sufficienti a tale scopo, è necessario dunque inserire un ulteriore metodo che analizzi queste regioni in modo tale da produrre una valutazione il più possibile accurata. Come metodo di valutazione si è pensato di cercare uno strumento che fosse in grado di riconoscere i pattern comuni all’interno dei segmenti precedentemente individuati. La scelta è immediatamente ricaduta su una branca dell’intelligenza artificiale in rapido sviluppo, il machine learning. Esistono infatti opportuni modelli particolarmente adatti a questo problema. Tali modelli saranno approfonditi in seguito in questa trattazione, illustrandone le potenzialità e il funzionamento. L’impiego combinato di tecnologie di apprendimento automatico e di analisi/elaborazione delle immagini risulta quindi efficace per la creazione di modelli che producano, partendo da semplici immagini dell’occhio, una analisi accurata in modo automatico. Il caso d’uso preso in considerazione per lo studio e l’applicazione dei metodi di elaborazione di immagini e degli algoritmi di machine learning è quello dell’iridologia, una teoria medica non ancora validata. L’obiettivo finale vuole essere quello di confrontare i risultati ottenuti da un modello adeguatamente accurato, creato sulla base delle immagini dell’iride, con quelli sostenuti dell'iridologia al fine di confermare la veridicità o meno della suddetta teoria.
 
